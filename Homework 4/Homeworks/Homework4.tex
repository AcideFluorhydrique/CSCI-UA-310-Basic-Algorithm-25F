\lhead{\textbf{Basic Algorithms, Fall 2025 \\ CSCI-UA.0310-001}}
\chead{\Large{\textbf{Homework 4}}}
%%%%%%%%%%%%%%%%%%%%%%%%%%%%%%%%%%%%%%%
% ENTER NAME BELOW!
%%%%%%%%%%%%%%%%%%%%%%%%%%%%%%%%%%%%%%%
\rhead{\textbf{Professor Regev}\\\textbf{Name:} ~~~~~~~~~~~~~~~~~~~~~~~~~~~~~~~~~~~~~}
%REPLACE THE TILDES WITH YOUR NAME
\runningheadrule
\firstpageheadrule
\cfoot{}

\section*{Due October 6 (11:59 p.m.)}
\intro

\subsection*{1-0 List all your collaborators and sources: (\texorpdfstring{$-\infty$}{-∞} points if left blank)}
\vspace{0.75cm}



\section*{Problem 4-1 -- Permutations (40 Points)}

Let $[n] := \{1, 2, \ldots, n\}$, and let $S_n$ be the set of all permutations of $[n]$. For example, for $n=3$, we have $|S_3|=6$ with:
$$S_3 = \{(1,2,3), (1,3,2), (2, 1,3), (2, 3, 1), (3, 1, 2), (3, 2, 1)\}$$
%Recall that $|S_n|=n!=n\cdot (n-1) \cdots 1$. %Recall that $n!=O(n^n)$ and $2^n=O(n!)$. 
Note that each permutation in $S_n$ can be seen as an input for a sorting algorithm (where the elements of $[n]$ are reordered according to the permutation). In this question, we will consider sorting algorithms that for some inputs err and do not return the correct answer. We say that a sorting algorithm is $\eps$-correct if the algorithm produces the correct result (i.e., produces a sorted array as output) on at least $\eps\cdot |S_n|$ inputs in the set $S_n$. 

\begin{enumerate}
    \item (25 points) Argue that for any $0 \leq \epsilon \leq 1$, the decision tree of an $\epsilon$-correct comparison-based sorting algorithm must have at least $\epsilon \cdot |S_n|$ leaves. \hint{Recall that the output of a comparison-based sorting algorithm is a permutation that prescribes how the input array has to be ordered to obtain the sorted output.}
    \begin{solution}
        \vspace{\stretch{1}}
        \vspace{0.5 cm}


        Let $S_n$ be the set of all permutations of $[n]$, 
        
        we have 
        \[|S_n| = n!\]
        

        \vspace{0.5 cm}
        
        Let $L$ be the number of leaves in its decision tree
        
        For each leaf $\ell$, the algorithm outputs a fixed permutation $\sigma_\ell$
        
        The algorithm is correct on an input permutation $p$ only if the output $\sigma_\ell$ equals $p^{-1}$, where $\ell$ is the leaf reached by $p$
        
        Since $\sigma_\ell$ is fixed, there is at most one $p$ per leaf for which the algorithm is correct
        
        Therefore, the number of input permutations on which the algorithm is correct is at most $L$
        
        Since the algorithm is $\epsilon$-correct, it must be correct on at least $\epsilon \cdot n!$ inputs
        
        Thus, \[L \geq \epsilon \cdot n!\]

        QED
        
    \end{solution}
\end{enumerate}

\noindent
In the following, we investigate whether allowing for more errors can save on the required number of comparisons.   
For example, at the extreme case that $\varepsilon =0$, no comparison has to be made (the algorithm is allowed to always err).
In what follows, for each given value of $\varepsilon$, decide whether there exists a comparison-based sorting algorithm that is $\varepsilon$-correct and runs in time $O(n)$.  Justify your answers. (Recall that $|S_n| = n! = n(n-1)\cdots2\cdot 1$.)
\begin{enumerate}[resume]
    \item (5 points) $\eps=1/2$. 
    
    \begin{solution}
        \vspace{\stretch{1}}
        \vspace{0.5 cm}

        For a comparison-based sorting algorithm to be $\epsilon$-correct with $\epsilon = 1/2$, the decision tree must have at least $\frac{1}{2} n!$ leaves
        
        However, if the algorithm runs in $O(n)$ time, 
        
        \hspace{1 cm} the height of the decision tree is $O(n)$
        
        \hspace{1 cm} so the maximum number of leaves is $2^{O(n)}$
        
        For large $n$, $2^{cn} < \frac{1}{2} n!$, which is impossible
        
        Hence, no such algorithm exist

        
    \end{solution}

    \item (5 points) $\eps=1/n$. 
    \begin{solution}
        \vspace{\stretch{1}}
        \vspace{0.5 cm}


        For a comparison-based sorting algorithm to be $\epsilon$-correct with $\epsilon = 1/n$, the decision tree must have at least \[(1/n) \cdot n! = (n-1)!\] leaves

        \vspace{0.5 cm}

        
        If the algorithm runs in $O(n)$ time, 
        
        \hspace{1 cm}the height of the decision tree is $O(n)$
        
        \hspace{1 cm}so the maximum number of leaves is $2^{O(n)}$
        
        For large $n$, $2^{O(n)} < (n-1)!$, which contradicts the requirement
        
        Therefore, no such algorithm exists

        
    \end{solution}
    
    \item (5 points) $\eps=\frac{1}{2^n}$. 
    \begin{solution}
        \vspace{\stretch{1}}
        
        \vspace{0.5 cm}

        For a comparison-based sorting algorithm to be $\epsilon$-correct with $\epsilon = 1/2^n$, the decision tree must have at least $(1/2^n) \cdot n!$ leaves
        
        If the algorithm runs in $O(n)$ time, 
        
        \hspace{1 cm}the height of the decision tree is $O(n)$
        
        \hspace{1 cm}so the maximum number of leaves is $2^{O(n)}$
        
        For very large $n$, we know exponential is much smaller than factorial:

        \[ k^n << n!\]

        Thus,
        

        \[2^{cn+b+1}< n!\]

        i.e.

        \[2^{\mathcal{O}(n) } < \frac{n!}{2^n}\]

     
        contradiction!
        
        Hence, no such algorithm exists
        
    \end{solution}

    % \newpage 
    
    \item (\textbf{Honors Problem}, **, 0 points)  $ \eps=\frac{2^{\left\lfloor \frac{n}{2}\right\rfloor }}{n!}$.
    \begin{solution}
        \vspace{\stretch{1}}
        \vspace{0.5 cm}

        For a comparison-based sorting algorithm to be $\epsilon$-correct with $\epsilon = \frac{2^{\lfloor n/2 \rfloor}}{n!}$
        
        the decision tree must have at least $2^{\lfloor n/2 \rfloor}$ leaves
        
        Since the algorithm runs in $\mathcal{O}(n)$ time, the decision tree has height $O(n)$, and thus the number of leaves is at most $2^{O(n)}$
        
        For large $n$, $2^{O(n)} > 2^{\lfloor n/2 \rfloor}$, so it is possible for the decision tree to have enough leaves to cover the required number of correct inputs
        
        Therefore, such an algorithm exists

        
    \end{solution}
    
\end{enumerate}

\section*{Problem 4-2 -- Sorting Bounds (30 points)}

\begin{enumerate}
    \item (5 points) Assume you are given an array of $n$ integers in the range
    $\{1,\ldots, (\log n)^{\log n}\}$. Show how to sort this array in time
    $O(n \log\!\log n)$.

    \begin{solution}
        \vspace{\stretch{1}}
        \vspace{0.5 cm}

        By lecture 8:

        If the keys are in range $(0, k^d-1)$,

        % By using $\mathcal{ RADIXSORT}$

        By using RADIXSORT, the runtime complexity would be 

        \[\Theta(d\cdot (n+k))\]

        Here, 

        \[(\log n)^{\log n} = n^{\log_n (\log n)^{\log n}} = n^{\log \log n}\]

        Hence, sorting this array would be 

        \[\Theta(\log \log n \cdot (n+n)) = \Theta(n \log \log n) \] 


        


        
    \end{solution}


    \item (10 points) Assume you are given an array of $n$ integers with many
    duplicates, so that you know that there are at most $\frac{n}{\log n}$
    distinct elements in the array. Show how to sort this array in time
    $O(n)$. You may assume access to an empty array of size $B$ where $B$ is an upper bound on the integers in your input as well as any additional memory you may require.
    
    \noindent(\textbf{Hint:} Adapt counting (small-key) sort by keeping track of which elements exist in the array and how many from each one.)

    \begin{solution}
        \vspace{\stretch{1}}
        \vspace{0.6 cm}

    Since there are at most $\frac{n}{\log n}$ distinct elements in the array, we can sort it in $O(n)$ time as follows:

    \begin{enumerate}
        \item Traverse the input array and use a hash map to count the frequency of each distinct element
        
        This takes $\mathcal{O}(n)$ time
        
        \item Extract all distinct keys from the hash map into a list 
        
        The size of this list is at most $\frac{n}{\log n}$
        
        \item Sort this list of distinct elements using any comparison-based sorting algorithm
        
        The time complexity is
        \[
            \mathcal{O}\left( \frac{n}{\log n} \cdot \log\left( \frac{n}{\log n} \right) \right)
            = \mathcal{O}\left( \frac{n}{\log n} \cdot (\log n - \log \log n) \right)
            = \mathcal{O}(n)
        \]
        
        \item Finally, iterate through the sorted list of distinct elements, and for each element, output it as many times as its frequency into the output array. This step also takes $\mathcal{O}(n)$ time
        
    \end{enumerate}
    
    Thus, the total running time is $$\mathcal{O}(n)$$

        
    \end{solution}

    \item (10 points) From class, we saw that comparison-based sorting algorithms must make $\Omega(n\log n)$ comparisons, and comparison-based median finding algorithms must take $\Omega(n)$ comparisons. For this question, we want to show that sorting takes at least as many explicit (i.e., not asymptotic) comparisons as median finding. 

    To do so, we must introduce the notion of a reduction. Suppose that solving some problem \textsf{probA} always requires at least $A$ comparisons in the worst case. Also suppose there is an algorithm for solving \textsf{probA} that works by solving a problem \textsf{probB} and then making no other comparisons before outputting. Prove that this implies that solving \textsf{probB} also takes at least $A$ comparisons. (\textbf{Hint:} Think about what happens if this weren't true.)
     \begin{solution} 
        \vspace{\stretch{1}}
        \vspace{0.6 cm}

    
    Suppose that solving problem \textsf{probA} requires at least $A$ comparisons in the worst case
    
    Assume there exists an algorithm $\mathcal{A}$ that solves \textsf{probA} by first solving \textsf{probB} and then producing the output without performing any additional comparisons

    Now, suppose for contradiction that \textsf{probB} can be solved using fewer than $A$ comparisons in the worst case
    
    Then, algorithm $\mathcal{A}$ would also use fewer than $A$ comparisons (since it only uses the comparisons made while solving \textsf{probB} and no others)
    
    This contradicts the assumption that any algorithm for \textsf{probA} must make at least $A$ comparisons in the worst case

    Therefore, our supposition is false, and solving \textsf{probB} must require at least $A$ comparisons in the worst case
    
    \end{solution} 

    \item (5 points) Using the above, prove that any comparison-based sorting algorithm must make at least as many comparisons as it takes to find the median of an array using a comparison-based algorithm. 
    \begin{solution}
        \vspace{0.5cm}

    Consider the problem of finding the median of an array
    
    We can solve this problem by first sorting the array using any comparison-based sorting algorithm and then simply returning the middle element (e.g., the $\lfloor n/2 \rfloor$-th element in the sorted order)
    
    This procedure makes no comparisons beyond those performed by the sorting algorithm itself

    Thus, we have reduced the median-finding problem (\textsf{probA}) to the sorting problem (\textsf{probB}) without any additional comparisons
    
    By the result from the previous part, this implies that the number of comparisons required to sort must be at least as large as the number required to find the median 
    
    In other words, any comparison-based sorting algorithm must make at least as many comparisons as any comparison-based median-finding algorithm
        
    \end{solution}
   
\end{enumerate}

% \newpage

\section*{Problem 4-4 -- 3-Way Karatsuba (30 Points)}

In this question, we will consider the problem of making 3-way Karatsuba multiplication. In essence, we will divide our numbers $X$ and $Y$ into three parts rather than 2, as so:

$$X = A\cdot10^{2n/3}+B\cdot10^{n/3}+C$$

$$Y=D\cdot 10^{2n/3}+E\cdot 10^{n/3}+F.$$
\begin{enumerate}
    \item (10 points) Describe the naive approach using 9 small-part multiplications to find the product $X\cdot Y$ using the split as above. 

    \begin{solution}
        \vspace{0.5cm}

        By writing
        
        \[X=A\,10^{2n/3}+B\,10^{n/3}+C\]
        
        \[ Y=D\,10^{2n/3}+E\,10^{n/3}+F\]
        
        we perform the nine small multiplications
        \[AD,\ AE,\ AF,\ BD,\ BE,\ BF,\ CD,\ CE,\ CF,\]
        and align them according to their place weights
        
        
        Grouping equal
        shifts gives
        \[
        \begin{aligned}
        XY
        &= (AD)\,10^{4n/3}
         + (AE+BD)\,10^{n}
         + (AF+BE+CD)\,10^{2n/3}\\
        &\quad + (BF+CE)\,10^{n/3}
         + (CF)
        \end{aligned}
        \]
    

        
    \end{solution}


    
    \item (10 points) Show how to reduce this number to 6 multiplications using interpolation similar to Karatsuba. 

    \begin{solution}
        \vspace{0.5 cm}
        
        Def
        \[P(x) = A x^2 + B x + C\]
        \[Q(x) = D x^2 + E x + F\]
        where $x = 10^{n/3}$
        
        % The goal is to compute the product $R(x) = P(x)Q(x)$, whose coefficients correspond to the terms of the final result
        
        % Direct expansion requires 9 multiplications, but through interpolation techniques, it can be reduced to 6
        Let $R(x) = P(x)\cdot Q(x)$
        
        Def
        \begin{align*}
            m_1 &= A \cdot D \\
            m_2 &= C \cdot F \\
            m_3 &= (A + C) \cdot (D + F) \\
            m_4 &= (A + B) \cdot (D + E) \\
            m_5 &= (B + C) \cdot (E + F) \\
            m_6 &= (A + B + C) \cdot (D + E + F)
        \end{align*}
    
        These products correspond to evaluations of the polynomials at specific points, 
        
        
        And through algebraic combinations, we can derive the coefficients of $R(x)$:
        \begin{align*}
            a &= m_1 \\
            b &= m_6 - m_5 + m_2 - m_3 \\
            c &= 2m_3 + m_4 + m_5 - 2m_1 - 2m_2 - m_6 \\
            d &= m_1 - m_4 + m_6 - m_3 \\
            e &= m_2
        \end{align*}
    

        
        $$
        X \cdot Y = a \cdot 10^{4n/3} + b \cdot 10^n + c \cdot 10^{2n/3} + d \cdot 10^{n/3} + e
        $$
        

    
    \end{solution}

    
    % \newpage

    \item (10 points) Describe the recurrence relation to find the time complexity of the three-way Karatsuba, and give its asymptotic complexity.

    \begin{solution}
        \vspace{0.5cm}

        From (2), we have,
        $$T(n) = 6T(n/3)+\mathcal{O}(n)$$
        
        $$T(n/3) = 6T(n/9) + c(n/3)$$
        
        \[
        T(n) = 6\left[6T(n/9) + c\cdot\frac{n}{3}\right] + cn = 6^2 T(n/9) + 2cn + cn
        \]
        
        \[
        T(n) = 6^2\left[6T(n/27) + c\cdot\frac{n}{9}\right] + cn(1 + 2) = 6^3 T(n/27) + 4cn + cn(1 + 2)
        \]
        
        After three steps:
        \[
        T(n) = 6^3 T(n/27) + cn(1 + 2 + 4)
        \]
        
        After $k$ iterations:
        
        \[
        T(n) = 6^k T\left(\frac{n}{3^k}\right) + cn \sum_{i=0}^{k-1} 2^i
        \]
        
        The geometric sum evaluates to:
        \[
        \sum_{i=0}^{k-1} 2^i = 2^k - 1
        \]
        
        Thus:
        \[
        T(n) = 6^k T\left(\frac{n}{3^k}\right) + cn(2^k - 1)
        \]
        
        We stop when the argument of $T$ becomes 1:
        
        \[
        \frac{n}{3^k} = 1 \quad \Rightarrow \quad k = \log_3 n
        \]
        
        Substitute $k = \log_3 n$
        
        \[
        T(n) = 6^{\log_3 n} T(1) + cn\left(2^{\log_3 n} - 1\right)
        \]
        
        Using the identity $a^{\log_b c} = c^{\log_b a}$
        
        \[
        6^{\log_3 n} = n^{\log_3 6}, \quad 2^{\log_3 n} = n^{\log_3 2}
        \]
        
        So
        
        \[
        T(n) = d \cdot n^{\log_3 6} + c n \cdot n^{\log_3 2} - cn = d \cdot n^{\log_3 6} + c n^{1 + \log_3 2} - cn
        \]
        
        Note that:
        
        \[
        1 + \log_3 2 = \log_3 3 + \log_3 2 = \log_3 6
        \]
        
        Hence:
        \[
        T(n) = (d + c) n^{\log_3 6} - cn
        \]
                
        Since $\log_3 6 \approx 1.631 > 1$, the term $n^{\log_3 6}$ dominates, and we conclude:
        \[
        T(n) = \Theta(n^{\log_3 6})
        \]

    \end{solution}

    \item (Honors Problem, *, 0 points)) Compare the time complexity derived above to if you were to use 2-way Karatsuba to find the product. Under what conditions, if any, is the 3-way version faster?

    \begin{solution}
        \vspace{4cm}
    \end{solution}
\end{enumerate}